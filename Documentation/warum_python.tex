\chapter{Warum Python?}

Viele Informatik-Kurse oder Vorlesungen f�r Anf�nger benutzen die Programmiersprache Python als erste 
Programmiersprache. Von den 39 besten Einf�hrungskursen f�r Informatik in den USA verwendeten im Jahr 2014 
27 Kurse Python als erste Programmiersprache. ~\cite{Guo.07.07.2014} Mit 69\% ist Python somit, mit einer 
eindeutigen Mehrheit die meist verwendete Programmiersprache unter diesen Kursen. 
\begin{quotation}
For the fifth year in row, Python retains it's \#1 dominance followed by Java, C++ and Javascript.
\end{quotation}\cite{codeeval.}\footnote{(Stand 01.08.2016)} 
Trotz der sinkenden Beliebtheit von Python auf der Seite code eval, ist Python dennoch seit Jahren die beliebteste 
Programmiersprache nach ihren Messungen. Ein weiterer Artikel, der Python als sehr beliebte Sprache darstellt und 
eine gro�e Relevanz f�r diese Arbeit tr�gt, da er die Verwendung von Programmiersprachen in Lehrveranstaltungen
beleuchtet, ist der Blogbeitrag f�r die Association for Computing Machinery (ACM) von Philip Guo.\footnote{
\url{http://cacm.acm.org/blogs/blog-cacm/176450-python-is-now-the-most-popular-introductory-teaching-language-at-top-us-universities/fulltext}
 (Stand 09.05.2016)} In dem Beitrag wird beschrieben, dass Python Java als h�ufigste Programmiersprache f�r Anf�nger abgel�st hat. 
Auch wenn der Artikel bereits 2014 ver�ffentlicht wurde, l�sst sich vermuten, dass die Verbreitung von Python nicht zur�ckgegangen 
ist. Grund hierf�r ist die steigende Beliebtheit der Sprache nach dem TIOBE Index\footnote{
\url{http://www.tiobe.com/tiobe_index} (Stand 09.05.2016)}, wie auch ein f�nfter Platz in der Statistik 
von Coding Dojo\footnote{\url{http://www.codingdojo.com/blog/9-most-in-demand-programming-languages-of-2016/} (Stand 09.05.2016)}.

Die Online-Lernplattform Udacity verwendet f�r den Kurs �Intro to Computer Science� Python als Sprache, 
um die Themen der theoretischen Informatik zu erl�utern. Diesen Online-Kurs haben bereits �ber 500.000 
Personen besucht.\footnote{Stand 09.05.2016} Als Proargumente werden die M�chtigkeit, 
die leichte Erlernbarkeit und die weite Verbreitung aufgef�hrt.

Zudem kommt, dass in \textsc{SetlX} nur Java-Bibliotheken eingebunden werden k�nnen. 
Selbst wenn eine Java-Bibliothek eingebunden wird, so muss diese direkt in der Programmiersprache 
eingebunden werden. Es ist nicht m�glich Bibliotheken in einzelne Programme einzubinden. 
Au�erdem existieren bereits einige Python-Bibliotheken, die der theoretischen Informatik Vorlesung 
einen Mehrwert bieten w�rden. Beispielsweise w�ren die Module matplotlib, zum Plotten von Graphen, 
sowie pandas, zur Auswertung von Daten, f�r Python vorhanden. Zwar kann in \textsc{SetlX} bereits 
geplottet werden, jedoch ist der Dozent der Vorlesung mit der Implementierung der Java-Bibliothek unzufrieden. 
