\chapter{Warum Python?}

Viele Informatik-Kurse oder Vorlesungen f�r Anf�nger benutzen die Programmiersprache Python als erste 
Programmiersprache. Von den 39 besten Einf�hrungskursen f�r Informatik in den USA verwendeten im Jahr 2014 
27 Kurse Python als erste Programmiersprache. ~\cite{Guo14} Mit 69\% ist Python somit, mit einer 
eindeutigen Mehrheit die meist verwendete Programmiersprache unter diesen Kursen. Einige 
Internet-Artikel, die die Beliebtheit von heutigen Programmiersprachen beleuchten, referenzieren �fter 
den Blogbeitrag f�r die Association for Computing Machinery (ACM). In dem Beitrag wird beschrieben, dass 
Python Java als h�ufigste Programmiersprache f�r Anf�nger abgel�st hat. Auch wenn der Artikel bereits 
2014 ver�ffentlicht wurde, l�sst sich vermuten, dass die Verbreitung von Python nicht zur�ckgegangen 
ist. Grund hierf�r ist die steigende Beliebtheit der Sprache nach dem TIOBE Index\footnote{
\url{http://www.tiobe.com/tiobe_index} (Stand 09.05.2016)}, wie auch ein f�nfter Platz in der Statistik 
von Coding Dojo\footnote{
\url{http://www.codingdojo.com/blog/9-most-in-demand-programming-languages-of-2016/} (Stand 09.05.2016)}.

Die Online-Lernplattform Udacity verwendet f�r den Kurs �Intro to Computer Science� Python als Sprache, 
um die Themen der theoretischen Informatik zu erl�utern. Diesen Online-Kurs haben bereits �ber 500.000 
Personen besucht.\footnote{Stand 09.05.2016} Als Proargumente werden die M�chtigkeit, die leichte Erlernbarkeit und die weite Verbreitung aufgef�hrt.