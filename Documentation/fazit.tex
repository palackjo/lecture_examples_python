\chapter{Fazit}
Prinzipiell kann gesagt werden, dass in den meisten F�llen die \textsc{SetlX}-Programme, mit Hilfe des 
lecture-Moduls, in Python �hnlich programmiert werden k�nnen. Die Syntax beider Sprachen unterscheidet 
sich stellenweise und manchmal muss ein Ausdruck nach einem anderen Schema aufgebaut werden. 
Jedoch bieten beide Sprachen f�r viele Aufgaben geeignete Methoden die eine einfache Probleml�sung erm�glichen. 

Eine Umstellung der Programmiersprache in der Vorlesung w�re durchaus m�glich. Die Performance-Differenzen, 
die in zwei der umgesetzten Implementierungen aufgetreten sind, sind zwar �rgerlich, 
jedoch befinden sich die Laufzeiten noch in akzeptablen Bereichen. Der Wechsel der Programmiersprache 
w�rde einer der Intentionen von \textsc{SetlX}, den Studenten eine Sprache n�herbringen, die eine sehr 
�hnliche Syntax wie die Sprachen Java und C besitzt,  nicht mehr verfolgen. Allerdings w�re Python 
eine st�rker verbreitete Programmiersprache und die Studenten w�rden somit auch die m�glichen 
Unterschiede zwischen Programmiersprachen erkennen. Das w�re auch eine gute Vorbereitung auf die Praxis, 
da dort h�ufig eine Vielzahl an verschiedenen Programmiersprachen Einsatz findet. Ein weiteres Problem w�re, 
dass die sehr mathematische Notation von \textsc{SetlX} nicht mehr vorhanden w�re. In Python werden 
h�ufig W�rter als Operatoren anstatt mathematischen Operatoren verwendet. Im Gegenzug dazu k�nnten sich Anf�nger, 
die noch nie Kontakt mit einer Programmiersprache den Code leichter lesen, da anfangs weniger mathematische 
Operatoren gelernt werden m�ssen um den Code verstehen zu k�nnen.
