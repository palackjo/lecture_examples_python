\chapter{Einleitung}

In der Vorlesung �Grundlagen und Logik� des Moduls Theoretische Informatik I f�hrt der Dozent 
Prof. Dr. Karl Stroetmann die Programmiersprache \textsc{SetlX} ein. \textsc{SetlX} ist eine auf Java 
basierende Sprache, die sehr gut geeignet ist, um den Pseudocode aus Vorlesungen ausf�hrbar zu machen. 
Diese Programmiersprache wirbt damit, dass die Verwendung von Mengen und Listen sehr gut unterst�tzt 
wird. Au�erdem k�nnen Ausdr�cke aus der Mengenlehre, so wie andere mathematischen Ausdr�cke in einer 
Syntax, die sehr �hnlich zur mathematischen Notation ist, implementiert werden. \cite{Stroetmann.30.07.2015} 
Da diese Vorlesung bereits im ersten Semester stattfindet und die Studenten parallel dazu eine 
Vorlesung aus dem Modul Mathematik I besuchen, k�nnen die Studenten Themen wie beispielsweise die 
Mengenlehre schneller kennen lernen. Die komplement�re Auseinandersetzung mit �hnlichen bis gleichen 
Themen in beiden Vorlesungen erm�glicht das gleichzeitige Lernen f�r zwei Vorlesungen.

Ein weiterer Vorteil f�r die Studenten ist, dass die Syntax von \textsc{SetlX}, zus�tzlich zum sehr 
mathematischen Stil, auch starke �hnlichkeiten zur Programmiersprache C aufweist. Selbst f�r die 
Studenten, die zuvor keinen Kontakt mit C hatten, ist das ein gro�er Vorteil, da im ersten Semester 
parallel zur theoretischen Informatik Vorlesung auch eine Vorlesung mit dem Titel �Programmieren in C� 
besucht werden muss. So muss kein starkes Umdenken stattfinden, wenn von \textsc{SetlX} zu C und auch 
umgekehrt gewechselt wird.

Die Hauptintention dieser Arbeit ist es, zu pr�fen, ob es m�glich w�re die \textsc{SetlX}-Programme, 
die in der Vorlesung gezeigt werden, in Python-Skripte zu �bersetzen. Ziel dabei ist es, 
die Eleganz der Programme beizubehalten, damit die Studenten die verschiedenen 
Algorithmen besser verstehen und erlernen k�nnen. Auch wenn es in Python teilweise 
m�glich w�re die Problemstellungen �ber fertige Implementierungen, vermutlich mit weniger Codezeilen, 
zu l�sen, soll dennoch der Hauptfokus auf den Lernprozess in der theoretischen Informatik liegen.
Die Studenten sollen die Algorithmen verstehen und selber anwenden k�nnen.
