\chapter{Skripte ohne spezielle Module}

Es ist bereits m�glich, einige \textsc{SetlX}-Programme ohne zus�tzliche Module in Python 
anzufertigen. Diese Python-Skripte wurden als erstes angefertigt, um feststellen zu k�nnen, ob es m�glich ist 
Python Syntax zu verwenden, ohne die Eleganz des Codes zu verlieren.

Beim Schreiben von Python-Skripten muss, im Gegensatz zu \textsc{SetlX}, bei der Einr�ckung auf eine 
Besonderheit geachtet werden. In Python ist es m�glich, sogenannte unsichtbare Fehler zu erhalten, 
indem man eine inkonsistente Einr�ckung verwendet. Es ist m�glich sowohl mit Leerzeichen, 
als auch mit Tabulatoren die Einr�ckung zu gestalten. Was jedoch verboten ist, 
ist die Verwendung beider Arten gleichzeitig, da Python daraufhin auf einen Fehler st��t. 
Deshalb ist es wichtig bereits am Anfang festzulegen, ob mit Leerzeichen oder mit Tabulatoren einger�ckt wird. 
Alle Python-Skripte, die im Rahmen dieser Arbeit erstellt wurden, werden Leerzeichen zur Einr�ckung verwendet.

Das erste Codebeispiel aus dem Logik-Skript befasst sich mit der Berechnung einer Summe der Zahlen von 1 
bis zur eingegebenen Zahl. Dieses Programm l�sst sich auch nahezu identisch in Python abbilden. 
Das originale \textsc{SetlX} Programm verwendet hierf�r eine Menge, die die Zahlen von 1 bis zur 
eingegebenen Zahl n enth�lt. �ber den �+/�-Operator wird die Summe aller Zahlen in der Menge ermittelt 
und ausgegeben. 

\begin{figure}[!ht]
\centering
\begin{Verbatim}[ frame         = lines, 
                  framesep      = 0.3cm, 
                  firstnumber   = 1,
                  labelposition = bottomline,
                  numbers       = left,
                  numbersep     = -0.2cm,
                  xleftmargin   = 0.0cm,
                  xrightmargin  = 0.0cm,
                ]
    n := read("Type a natural number and press return: ");
    s := +/ { 1 .. n };
    print("The sum 1 + 2 + ... + ", n, " is equal to ", s, ".");
\end{Verbatim}
\vspace*{-0.3cm}
\caption{Einfache Summenberechnung in \textsc{SetlX}}
\label{fig:sum.setlx}
\end{figure}

In Python wurde fast dasselbe Verhalten nachgebildet. Jedoch wurde, statt einer Menge, 
eine Range der Zahlen von 0 bis n verwendet. Die Summe wird �ber die in Python bereits integrierte Funktion 
\texttt{sum()} berechnet und daraufhin ausgegeben. Prinzipiell findet derselbe Ablauf statt, 
allerdings werden unterschiedliche Datenstrukturen verwendet. Um alle Zahlen von Null bis n zu durchlaufen wird 
das Range Objekt in Python verwendet, hierbei ist zu beachten, dass das letzte Element (n+1) exklusiv ist. 
In \textsc{SetlX} hingegen wird direkt von Eins bis n iteriert. Bei der Ermittlung der Summe sind beide 
Ans�tze zul�ssig, da eine Addition mit Nullen keine �nderung bewirkt.

\begin{figure}[!ht]
\centering
\begin{Verbatim}[ frame         = lines, 
                  framesep      = 0.3cm, 
                  firstnumber   = 1,
                  labelposition = bottomline,
                  numbers       = left,
                  numbersep     = -0.2cm,
                  xleftmargin   = 0.0cm,
                  xrightmargin  = 0.0cm,
                ]
    n = int(input('Type a natural number and press return: '))
    s = sum(range(n + 1))
    print('The sum 1 + 2 + ... + ', n, ' is equal to ', s, '.')
\end{Verbatim}
\vspace*{-0.3cm}
\caption{Einfache Summenberechnung in Python}
\label{fig:sum.python}
\end{figure}

Allgemein kann gesagt werden, dass ein \textsc{SetlX}-Programm, ohne spezielle Funktionen oder 
Strukturen die nicht in Python wiedergefunden werden, meist �hnlich in Python nachgebildet werden kann.
