\chapter{Skripte ohne spezielle Module}

Trotz dessen, dass die Erstellung des Python Moduls als Hauptbestandteil dieser Arbeit gesehen wird, ist 
es durchaus m�glich einige \textsc{SetlX} Programme ohne zus�tzliche, nicht enthaltene Module 
anzufertigen. Diese Skripte wurden als erstes angefertigt, um feststellen zu k�nnen, ob es m�glich ist 
Python Syntax zu verwenden, ohne die Eleganz des Codes zu verlieren. Einer der Ziele der �bersetzung ist 
die Eleganz der Programme beizubehalten. 

Das erste Codebeispiel aus dem Logik-Skript befasst sich mit der Berechnung einer Summe der Zahlen von 1 
bis zur eingegebenen Zahl. Dieses Programm l�sst sich auch nahezu eins-zu-eins so in Python abbilden. 
Das originale \textsc{SetlX} Programm verwendet hierf�r eine Menge, die die Zahlen von 1 bis zur 
eingegebenen Zahl n enth�lt. Daraufhin wird die Summe aller in der Menge enthaltenen Zahlen mit dem �+/�-
Operator ermittelt und ausgegeben. 
\begin{figure}[!ht]
\centering
\begin{Verbatim}[ frame         = lines, 
                  framesep      = 0.3cm, 
                  firstnumber   = 1,
                  labelposition = bottomline,
                  numbers       = left,
                  numbersep     = -0.2cm,
                  xleftmargin   = 0.0cm,
                  xrightmargin  = 0.0cm,
                ]
    n := read("Type a natural number and press return: ");
    s := +/ { 1 .. n };
    print("The sum 1 + 2 + ... + ", n, " is equal to ", s, ".");
\end{Verbatim}
\vspace*{-0.3cm}
\caption{Einfache Summenberechnung in \textsc{SetlX}}
\label{fig:sum.setlx}
\end{figure}
In Python wurde fast dasselbe Verhalten nachgebildet. Jedoch wurde anstatt eine Menge anzufertigen eine 
Range der Zahlen von 0 bis n angelegt. Die Summe wird �ber die in Python bereits integrierte Funktion 
\texttt{sum()} berechnet und daraufhin ausgegeben.
\begin{figure}[!ht]
\centering
\begin{Verbatim}[ frame         = lines, 
                  framesep      = 0.3cm, 
                  firstnumber   = 1,
                  labelposition = bottomline,
                  numbers       = left,
                  numbersep     = -0.2cm,
                  xleftmargin   = 0.0cm,
                  xrightmargin  = 0.0cm,
                ]
    n = int(input('Type a natural number and press return: '))
    s = sum(range(n + 1))
    print('The sum 1 + 2 + ... + ', n, ' is equal to ', s, '.')
\end{Verbatim}
\vspace*{-0.3cm}
\caption{Einfache Summenberechnung in Python}
\label{fig:sum.python}
\end{figure}
Allgemein kann gesagt werden, dass ein \textsc{SetlX}-Programm, ohne spezielle Funktionen oder 
Strukturen die nicht in Python wiedergefunden werden, meist eine gro�e �hnlichkeit mit der Python-
Implementierung hat.
